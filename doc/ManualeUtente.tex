\documentclass{report}
\usepackage{hyperref} 
\usepackage{graphicx}

\title{Manuale Utente - Book Recommender}
\author{Mattia Papaccioli - 747053 - CO}

\begin{document}

\maketitle

\tableofcontents

\chapter{Installazione}
\section{Download eseguibile}
Per eseguire il programma Book Recommender e necessario scaricarlo al seguente link \url{https://github.com/sbOogway/lab_book/releases/tag/v1.0}. 
Selezionare il client in base al proprio sistema operativo. 
Infine eseguirlo con doppio click o con i comandi seguenti

\subsection{Server}
\texttt{java -jar server-1.0.jar}

\subsection{Client}
\texttt{java -jar client-1.0-linux.jar}


\section{Building from source}
\subsection{Dependencies}
Sono richiesti git maven e java per buildare l applicazione.
\url{https://maven.apache.org/install.html}
\subsection{Commands}
Clonare la repo. \\ \\
\verb+git clone https://github.com/sbOogway/lab_book+ \\ \\
Build it. \\ \\
\verb+cd lab_book+ \\ \\
\verb+mvn clean install+ \\ \\
i file jar si trovano in \verb+client/target/client.jar+ e in \verb+server/target/server.jar+

\chapter{Esecuzione}
\section{Server}
Prima di eseguire il server e necessario creare un database postgres con il nome \verb+book+. 
Eseguire il server tramite il comando \\ \\ 
\verb+java -jar server.jar+\\ \\
all avvio inserire l indirizzo dell host del database, la porta del database, l username del database e la password del database. \\ \\

\includegraphics[width=\linewidth]{pics/server.png}

dopo aver lanciato il server viene esposto un servizio rmi sulla porta 1099.


\section{Client}
Per eseguire il client fare doppio click sul jar del client oppure eseguire il jar da linea di comando.

Per consultare libri nel database schiacciare il tasto query in alto a destra. \\ \\
\includegraphics[width=\linewidth]{pics/query.png} \\ \\
Per registrarsi all applicazione schiacciare il tasto signup in alto a destra. \\ \\
\includegraphics[width=\linewidth]{pics/signup.png} \\ \\
Per fare login schiacciare il tasto login in alto a destra. \\ \\
\includegraphics[width=\linewidth]{pics/login.png} \\ \\

\subsection{Logged in user}
Per vedere le proprie librerie schiacciare il tasto libraries in alto a destra. \\ \\ 
\includegraphics[width=\linewidth]{pics/libs.png} \\ \\
Per creare recensioni e suggerimenti schiacciare l apposito tasto di fianco ad ogni libro. \\ \\ 
\includegraphics[width=\linewidth]{pics/rev_and_sugg.png} \\ \\

\chapter{Limiti della soluzione sviluppata}
\section{No more books}
Massimo tre suggerimenti per libro. \\ \\
Massimo 256 caratteri per note nelle recensioni. \\ \\
Non e possibile aggiungere libri al database. \\ \\ % o forse si tramite sqlinjection ;) \\ \\
\section{Note}
Non tutti i libri creati nella storia sono presenti nel database. 
fonte libri => \url{https://www.kaggle.com/datasets/elvinrustam/books-dataset/data}
\end{document}