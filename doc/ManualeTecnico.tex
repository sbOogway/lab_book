\documentclass{report}
\usepackage{hyperref} 
\usepackage{graphicx}
\usepackage[font=small,labelfont=bf]{caption}
% \usepackage[a4paper, right=2cm, left=2cm, top=2cm]{geometry} 
\usepackage{listings}
\usepackage{color}
\lstset{
    basicstyle=\footnotesize\ttfamily,
    numbers=left,
    numberstyle=\tiny\color{white},
    stepnumber=1,
    numbersep=10pt,
    breaklines=true,
    frame=single,
    language=SQL,
    showstringspaces=false
}
\title{Manuale Utente - Book Recommender}
\author{Mattia Papaccioli - 747053 - CO}

\begin{document}

\maketitle

\tableofcontents


\chapter{Struttura}
\section{Descrizione}
L'applicazione ha un archittetura client server basata su java rmi. Questa struttura permette gestione della concorrenza e distribuzione del servizio. Il database puo essere esterno e non e necessario che sia hostato sulla stessa macchina del server. Senza un server rmi attivo il client non puo eseguire. E necessaria quindi una connesione di rete per eseguire il programma.

\chapter{Database}
\begin{center}
\includegraphics[scale=0.4]{pics/er.png} 
\captionof{figure}{er schema} 
\end{center}

Questo documento descrive in dettaglio lo script SQL utilizzato per la creazione e inizializzazione del database dedicato alla gestione di:
\begin{itemize}
    \item Utenti registrati
    \item Librerie personali
    \item Libri
    \item Valutazioni dei libri
    \item Suggerimenti di libri
\end{itemize}

Lo script implementa la struttura relazionale del database definendo le tabelle, i relativi vincoli di integrità (chiavi primarie, chiavi esterne, vincoli di unicità e vincoli \texttt{CHECK}) e indici per ottimizzare le operazioni di ricerca.

\section{Descrizione delle Tabelle}

\subsection{Tabella \texttt{UtentiRegistrati}}
Questa tabella memorizza le informazioni degli utenti registrati.

\begin{lstlisting}
CREATE TABLE UtentiRegistrati (
    id SERIAL PRIMARY KEY,
    userid VARCHAR(100) UNIQUE NOT NULL,
    nome VARCHAR(100) NOT NULL,
    cognome VARCHAR(100) NOT NULL,
    codice_fiscale VARCHAR(16) UNIQUE NOT NULL,
    email VARCHAR(100) UNIQUE NOT NULL,
    password VARCHAR(255) NOT NULL
);
\end{lstlisting}

\begin{itemize}
    \item \textbf{id}: Identificativo univoco, generato automaticamente tramite il tipo \texttt{SERIAL}.
    \item \textbf{userid}: Nome utente univoco, vincolato da \texttt{UNIQUE} e non nullo.
    \item \textbf{nome} e \textbf{cognome}: Dati anagrafici dell'utente, entrambe obbligatorie.
    \item \textbf{codice\_fiscale} ed \textbf{email}: Campi univoci per ogni utente, vincolati da \texttt{UNIQUE} e non nulli.
    \item \textbf{password}: Memorizza la password dell'utente, con una lunghezza massima di 255 caratteri.
\end{itemize}

\subsection{Tabella \texttt{Librerie}}
Questa tabella gestisce le librerie personali degli utenti.

\begin{lstlisting}
CREATE TABLE Librerie (
    id SERIAL PRIMARY KEY,
    nome_libreria VARCHAR(100) NOT NULL,
    userid INT NOT NULL,
    libri INT[],
    FOREIGN KEY (userid) REFERENCES UtentiRegistrati(id) ON DELETE CASCADE
);
\end{lstlisting}

\begin{itemize}
    \item \textbf{id}: Identificativo univoco della libreria.
    \item \textbf{nome\_libreria}: Nome della libreria, campo obbligatorio.
    \item \textbf{userid}: Riferimento all'utente proprietario della libreria. La chiave esterna punta a \texttt{UtentiRegistrati(id)} con eliminazione a cascata (\texttt{ON DELETE CASCADE}).
    \item \textbf{libri}: Array di interi (\texttt{INT[]}) contenente gli identificativi dei libri presenti nella libreria.
\end{itemize}

\subsection{Tabella \texttt{Libri}}
Questa tabella memorizza le informazioni relative ai libri.

\begin{lstlisting}
CREATE TABLE Libri (
    id SERIAL PRIMARY KEY,
    titolo VARCHAR(255) NOT NULL,
    autore VARCHAR(255) NOT NULL,
    editore VARCHAR(255),
    categoria TEXT[],
    anno INT CHECK (anno > 0)
);
\end{lstlisting}

\begin{itemize}
    \item \textbf{id}: Identificativo univoco del libro.
    \item \textbf{titolo} e \textbf{autore}: Informazioni fondamentali del libro, campi obbligatori.
    \item \textbf{editore}: Campo opzionale per l'editore.
    \item \textbf{categoria}: Array di categorie in cui il libro è classificato.
    \item \textbf{anno}: Anno di pubblicazione, con un vincolo \texttt{CHECK} che assicura il valore positivo.
\end{itemize}

\subsection{Tabella \texttt{ValutazioniLibri}}
Questa tabella registra le valutazioni assegnate dagli utenti ai libri.

\begin{lstlisting}
CREATE TABLE ValutazioniLibri (
    id SERIAL PRIMARY KEY,
    libro_id INT NOT NULL,
    utente_id INT NOT NULL,
    stile INT CHECK (stile BETWEEN 1 AND 5),
    contenuto INT CHECK (contenuto BETWEEN 1 AND 5),
    gradevolezza INT CHECK (gradevolezza BETWEEN 1 AND 5),
    originalita INT CHECK (originalita BETWEEN 1 AND 5),
    edizione INT CHECK (edizione BETWEEN 1 AND 5),
    voto_finale INT CHECK (voto_finale BETWEEN 1 AND 5),
    note VARCHAR(256),
    FOREIGN KEY (libro_id) REFERENCES Libri(id) ON DELETE CASCADE,
    FOREIGN KEY (utente_id) REFERENCES UtentiRegistrati(id) ON DELETE CASCADE
);
\end{lstlisting}

\begin{itemize}
    \item \textbf{id}: Identificativo univoco della valutazione.
    \item \textbf{libro\_id}: Riferimento al libro valutato, con chiave esterna verso \texttt{Libri(id)}. L'eliminazione a cascata garantisce la coerenza dei dati.
    \item \textbf{utente\_id}: Riferimento all'utente che ha effettuato la valutazione, collegato a \texttt{UtentiRegistrati(id)}.
    \item Vari campi (\texttt{stile, contenuto, gradevolezza, originalita, edizione, voto\_finale}) rappresentano le diverse metriche di valutazione, ciascuna con un vincolo \texttt{CHECK} che ne limita i valori compresi tra 1 e 5.
    \item \textbf{note}: Campo opzionale per eventuali commenti aggiuntivi.
\end{itemize}

\subsection{Tabella \texttt{ConsigliLibri}}
Questa tabella contiene i suggerimenti di lettura basati su un libro di riferimento.

\begin{lstlisting}
CREATE TABLE ConsigliLibri (
    id SERIAL PRIMARY KEY,
    libro_id INT NOT NULL,
    utente_id INT NOT NULL,
    libro1_suggerito_id INT NOT NULL,
    libro2_suggerito_id INT NOT NULL,
    libro3_suggerito_id INT NOT NULL,
    FOREIGN KEY (libro_id) REFERENCES Libri(id) ON DELETE CASCADE,
    FOREIGN KEY (utente_id) REFERENCES UtentiRegistrati(id) ON DELETE CASCADE,
    FOREIGN KEY (libro1_suggerito_id) REFERENCES Libri(id) ON DELETE CASCADE,
    FOREIGN KEY (libro2_suggerito_id) REFERENCES Libri(id) ON DELETE CASCADE,
    FOREIGN KEY (libro3_suggerito_id) REFERENCES Libri(id) ON DELETE CASCADE
);
\end{lstlisting}

\begin{itemize}
    \item \textbf{id}: Identificativo univoco del consiglio.
    \item \textbf{libro\_id}: Libro per il quale sono forniti i suggerimenti, chiave esterna verso \texttt{Libri(id)}.
    \item \textbf{utente\_id}: Utente destinatario del consiglio, chiave esterna verso \texttt{UtentiRegistrati(id)}.
    \item \textbf{libro1\_suggerito\_id, libro2\_suggerito\_id, libro3\_suggerito\_id}: Tre riferimenti a libri suggeriti, ognuno collegato alla tabella \texttt{Libri(id)}.
\end{itemize}

\section{Indici per il Miglioramento delle Performance}
Per ottimizzare le operazioni di ricerca e filtraggio, sono stati creati i seguenti indici:

\begin{lstlisting}
CREATE INDEX idx_libro_autore ON Libri(autore);
CREATE INDEX idx_valutazioni_libro ON ValutazioniLibri(libro_id);
CREATE INDEX idx_consigli_libro ON ConsigliLibri(libro_id);
\end{lstlisting}

\begin{itemize}
    \item \textbf{idx\_libro\_autore}: Agevola le query che filtrano per autore nella tabella \texttt{Libri}.
    \item \textbf{idx\_valutazioni\_libro}: Ottimizza l'accesso alle valutazioni di uno specifico libro nella tabella \texttt{ValutazioniLibri}.
    \item \textbf{idx\_consigli\_libro}: Favorisce le query che recuperano i consigli associati a un determinato libro.
\end{itemize}


\section{Considerazioni Finali}
\begin{itemize}
    \item L'uso delle clausole \texttt{ON DELETE CASCADE} garantisce la coerenza referenziale eliminando automaticamente i record figli in caso di cancellazione di un record padre.
    \item I vincoli \texttt{UNIQUE} sui campi critici (\texttt{userid}, \texttt{codice\_fiscale} e \texttt{email}) impediscono duplicazioni indesiderate.
    \item I vincoli \texttt{CHECK} in varie tabelle (per l'anno in \texttt{Libri} e per le valutazioni in \texttt{ValutazioniLibri}) assicurano l'integrità dei dati inseriti.
    \item L'organizzazione del database è stata progettata per sostenere operazioni CRUD efficienti e query complesse relative a libri, valutazioni e suggerimenti, supportata da indici strategici per l'ottimizzazione delle prestazioni.
\end{itemize}

Durante lo sviluppo del progetto, il database e stato modificato solo una volta cambiando 3 colonne per i suggerimenti con un array. 


\begin{center}
\includegraphics[scale=0.4]{pics/db_refactor.png} 
\captionof{figure}{git diff of database init script} 
\end{center}

\chapter{Uml}
\includegraphics[width=\linewidth]{pics/client_uml.jpg}
\captionof{figure}{Client uml} 
\includegraphics[width=\linewidth]{pics/common_uml.jpeg} 
\captionof{figure}{Common uml} 
\includegraphics[width=\linewidth]{pics/server_uml.jpg} 
\captionof{figure}{Server uml} 
\includegraphics[width=\linewidth]{pics/sequence.png} 
\captionof{figure}{Sequence diagram} 
\subsection{Modules}

\begin{itemize}
    \item \verb+client+ $ \Rightarrow $ Scritto in javafx, interagisce con il server per mostrare all utente la repository di libri e per salvare le proprie librerie. E comoposto da una \verb+classe+ main che lancia il programma e dal \verb+controller+ che gestisce la logica dell interazione dell utente. La struttura vera e propria dell'interfaccia e scritta in \verb+fxml+
    \item \verb+server+ $ \Rightarrow $ Si interfaccia con un database postgres per i dati ed espone un servizio rmi per l'interazione con il client.
    \item \verb+commom+ $ \Rightarrow $ Contiene metodi di utilità comuni a \verb+client+ e \verb+server+. Contiene inoltre le interfacce \verb+java rmi+.
\end{itemize}
Entrambi i moduli \verb+client+ e \verb+server+ utilizzano il modulo \verb+common+. Di particolare importanza sono le interfacce \verb+UsersInterface+ e \verb+Booksinterface+ che permettono la comunicazione tramite java rmi. Le classi \verb+Library+ e \verb+Review+ sono utilizzate per la serializzazione in java rmi e la classe \verb+Util+ contiene metodi utilitari.

\chapter{Pattern}
Viene utilizzato il pattern Client-Server per garantire controllo dei dati, distributività, concorrenza e modularita dell'applicazione.
Viene utilizzato il pattern MVC (Model View Controller). Il \verb+model+ è l'interazione da parte del server con il database, la \verb+view+ e rappresentata dal codice \verb+fxml+ e il \verb+controller+ e contenuto nel modulo \verb+client+. 
\end{document}