\documentclass{report}
\usepackage{hyperref} 
\usepackage{graphicx}
\usepackage[font=small,labelfont=bf]{caption}
% \usepackage[a4paper, right=2cm, left=2cm, top=2cm]{geometry} 

\title{Manuale Utente - Book Recommender}
\author{Mattia Papaccioli - 747053 - CO}

\begin{document}

\maketitle

\tableofcontents


\chapter{Struttura}
\section{Descrizione}
L'applicazione ha un archittetura client server basata su java rmi. Questa struttura permette gestione della concorrenza e distribuzione del servizio. Il database puo essere esterno e non e necessario che sia hostato sulla stessa macchina del server. Senza un server rmi attivo il client non puo eseguire.

\chapter{Database}
\includegraphics[width=\linewidth]{pics/er.png} 
\captionof{figure}{er schema} 

\chapter{Uml}
\includegraphics[width=\linewidth]{pics/client_uml.jpg}
\captionof{figure}{Client uml} 
\includegraphics[width=\linewidth]{pics/common_uml.jpeg} 
\captionof{figure}{Common uml} 
\includegraphics[width=\linewidth]{pics/server_uml.jpg} 
\captionof{figure}{Server uml} 
Entrambi i moduli \verb+client+ e \verb+server+ utilizzano il modulo \verb+common+. Di particolare importanza sono le interfacce \verb+UsersInterface+ e \verb+Booksinterface+ che permettono la comunicazione tramite java rmi. Le classi \verb+Library+ e \verb+Review+ sono utilizzate per la serializzazione in java rmi e la classe \verb+Util+ contiene metodi utilitari.
\end{document}